The abstract acts as a description of the reports contents. This allows for
the possibility to have a quick review of the report and provides an
overview of the whole report, i.e. contains everything from the
objectives and methods to the results and conclusions. Examples: “The
objective of this study has been to answer the question…. The study has
been conducted with the aid of…. The study has shown that…” Do not
mention anything that is not covered in the report. An abstract is written
as one piece and the recommended length is 200-250 words. References
to the report's text, sources or appendices are not allowed; the abstract
should “stand on its own”. Only use plain text, with no characters in
italic or boldface, and no mathematical formulas. The abstract can be
completed by the inclusion of keywords; this can ease the search for the
report in the library databases.\\
\newline
\textbf{Keywords:} Human-computer-interaction, XML, Linux, Java. 